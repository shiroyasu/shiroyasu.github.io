\documentclass[a4j,landscape,25pt]{jsarticle}

\usepackage{mypresen}
\usepackage[display]{texpower}

\def\thesection{\hspace{-1zw}}
\def\thesection{\hspace{-1zw}}

\AtBeginDvi{\special{papersize=297mm,210mm}}
\setcounter{page}{0}
\setlength{\headheight}{10pt}
\setlength{\textheight}{215pt}
\setlength{\footskip}{10pt}

\usepackage{fancyheadings}
\pagestyle{fancy}
\headrulewidth=0mm
\footrulewidth=0mm
\lhead{}
\rhead{}
\chead{}
\lfoot{}
\rfoot{{\footnotesize (\thepage)}}
\cfoot{}

\definecolor{gray1}{gray}{0.7}

\begin{document}

%%% タイトルページ
\thispagestyle{empty}
\vspace*{30pt}
\begin{center}
{\small 東京電機大学 情報環境学部}
\vspace{15pt}

{\large 2012年度 前学期\\数学科目のクラス分けについて}
\vspace{45pt}

{\footnotesize 
平成24年4月9日(月)\\
担当:佐藤 弘康
}
\end{center}

\section{数学科目について(概略)}
\noindent 第1セメスター:\fbox{\color{blue}{基礎数学}}\quad\fbox{\color{blue}{線形代数}}\hspace{30pt}{\small (注意:「↓」は\underline{事前履修条件})}

         ↓    \hspace{3.5pt}↓

\noindent 第2セメスター:\fbox{\color{blue}{微分積分学}}\hspace{4pt}\quad↓\hspace{37pt}\fbox{\color{blue}{確率統計}}\hspace{37pt}\fbox{計算機数学}

         ↓    \hspace{3pt}↓\hspace{150pt}↓

\noindent 第3セメスター:\fbox{情報数学I}\hspace{17pt}↓\hspace{17pt}\fbox{情報数学II}\quad\fbox{情報数学V}\hspace{17pt}↓

\hspace{132pt}↓\hspace{150.5pt}↓

\noindent 第4セメスター:     \fbox{情報数学III}\hspace{22pt}\fbox{情報数学IV}\hspace{22pt}\fbox{離散数学}
\medskip

\begin{itemize}
\item 上記の他に前学期:\underline{微分積分学(再履修)}
\item      後学期:\underline{基礎数学(再履修)},\underline{線形代数(再履修)}
\item 数学教員免許関連科目:\underline{数学科教育法},\underline{幾何学I},\underline{幾何学II},\underline{幾何学III}
\item 今年度は前学期にも\underline{情報数学III}を開講.
 \end{itemize}

\section{基礎数学}

\begin{itemize}
 \item 田澤先生(月水金 1時限)
 \item 原 先生(月水金 1時限)
 \item ヌルメメット先生(月水金 1時限)
 \item ヌルメメット先生(月水金 2時限)
\end{itemize}

\begin{itemize}
 \item 瀧 先生「基礎数学特別クラス」(月水金 2時限,水 4時限,金6時限)
\begin{itemize}
 \item 注意:このクラスの学生は
\underline{前期の「線形代数」履修不可}.
\end{itemize}
\end{itemize}

\begin{itemize}
 \item 原 先生「微分積分学」(月水金 2時限,金 7時限)
\begin{itemize}
 \item 注意:基礎数学の履修を免除.
 \item 金7時限は大無田先生が担当.
\end{itemize}
\end{itemize}

\section{線形代数}

\begin{itemize}
 \item 根本先生(月水金 4時限,金6時限)
 \item 根本先生(月水金 5時限,金6時限)
\begin{itemize}
 \item 金6時限は豊村先生が担当.
\end{itemize}
 \item 佐藤  (月水金 5時限,金6時限)
\end{itemize}

\begin{itemize}
 \item アハメド・アシュラフ先生「英語クラス」(月水金 4時限,金6時限)
\begin{itemize}
 \item 月 4時限に瀧先生が日本語で解説.
\end{itemize}
 \item アハメド・アシュラフ先生「英語クラス」(月水金 5時限,金7時限)
\begin{itemize}
 \item 月 5時限に山本先生が日本語で解説.
\end{itemize}
\end{itemize}

\section{注意事項}

\begin{itemize}
 \item 原則的に\underline{指定されたクラス}の授業をとってもらいます.
 \item クラス移動を希望する場合;\\
 「基礎数学特別クラス」から通常の「基礎数学」へ\\
 \quad $\Longrightarrow$田澤先生クラス(1時限)を推奨.\\
「線形代数英語クラス」から通常の「線形代数」へ\\
 \quad $\Longrightarrow$4時限クラスは根本先生クラス(4時限)へ\\
 \quad $\Longrightarrow$5時限クラスは根本先生クラス(5時限),または佐藤クラスへ\\
いずれの場合も,\underline{初回は指定されたクラスに出席}し,ガイダンスを受け,
担当教員の指示にしたがってください.\\
 \item 数学科目は英語科目とリンクしています.
受講時限が変わる場合は英語担当の先生にも必ず連絡してください.
\end{itemize}

\section{注意事項}

\begin{itemize}
 \item 「基礎数学」は「微分積分学」の事前履修条件になっているため,
「微分積分学」のクラスに割り当てられた学生は\underline{通常の方法で履修申告できま}{ }\underline{せん}.
履修を希望する学生はかならず初回授業に出席し,
担当教員の支持に従ってください.\\
 \item 編入学生は単位互換等について学級担任に相談してください.\\
 \item その他何かわからないことがあれば
\underline{佐藤}に相談してください.\\
(研究棟501教員室,hiroyasu@sie.dendai.ac.jp)
\end{itemize}

\end{document}