\documentclass[a4j,landscape,25pt]{jsarticle}

\usepackage{amsmath,amssymb,ascmac, bm, graphics}
\usepackage{mypresen}
\usepackage[display]{texpower}

\def\thesection{\hspace{-1zw}}
\def\thesection{\hspace{-1zw}}

\AtBeginDvi{\special{papersize=297mm,210mm}}
\setcounter{page}{0}
\setlength{\headheight}{10pt}
\setlength{\textheight}{215pt}
\setlength{\footskip}{10pt}

\usepackage{fancyheadings}
\pagestyle{fancy}
\headrulewidth=0mm
\footrulewidth=0mm
\lhead{}
\rhead{}
\chead{}
\lfoot{}
\rfoot{{\footnotesize (\thepage)}}
\cfoot{}

\newcommand{\defarrow}{\stackrel{\mathrm{\sf def}}{\Longleftrightarrow}}
\newcommand{\iffarrow}{\stackrel{\mathrm{\sf iff}}{\Longleftrightarrow}}
\newcommand{\isomeq}{\stackrel{\mathrm{\sf isom}}{=}}
\newcommand{\RN}{\mathbb{R}}
\newcommand{\CN}{\mathbb{C}}
\newcommand{\HN}{\mathbb{H}}
\newcommand{\ON}{\mathbb{O}}
\def\diff{\mathop{\mathrm{Diff}}\nolimits}
\def\vol{\mathop{\mathrm{Vol}}\nolimits}
\def\isom{\mathop{\mathrm{Isom}}\nolimits}
\def\id{\mathop{\mathrm{id}}\nolimits}
\def\grad{\mathop{\mathrm{grad}}\nolimits}
\def\trace{\mathop{\mathrm{trace}}\nolimits}

\definecolor{gray1}{gray}{0.7}

\begin{document}

%%% タイトルページ
\thispagestyle{empty}
\vspace*{30pt}
\begin{center}
{\small 東京電機大学 情報環境学部}
\vspace{15pt}

{\large 2011年度 前学期\\数学科目のクラス分けについて}
\vspace{45pt}

{\footnotesize 
平成23年4月15日(金)\\
担当:佐藤 弘康
}
\end{center}

\section{基礎数学}

\begin{itemize}
 \item 田澤先生(月水金 1時限)
 \item 原 先生(月水金 1時限)
 \item ヌルメメット先生(月水金 1時限)
 \item ヌルメメット先生(月水金 2時限)
\end{itemize}

\begin{itemize}
 \item 佐藤担当「基礎数学特別クラス」(月水金 2時限,水 4時限,金6時限)
\begin{itemize}
 \item このクラスの学生は
\underline{「線形代数」を履修することができません}.
\end{itemize}
\end{itemize}

\begin{itemize}
 \item 原 先生「微分積分学」(月水金 2時限,金 7時限)
\begin{itemize}
 \item 基礎数学の履修を免除.
 \item 金7時限は青山先生
\end{itemize}
\end{itemize}

\section{線形代数}

\begin{itemize}
 \item 根本先生(月水金 4時限,金6時限)
 \item 根本先生(月水金 5時限,金6時限)
\begin{itemize}
 \item 金6時限は豊村先生
\end{itemize}
 \item 瀧 先生(月水金 5時限,金6時限)
\end{itemize}

\begin{itemize}
 \item アハメド・アシュラフ先生「英語クラス」(月水金 4時限,金6時限)
\begin{itemize}
 \item 水 4時限に瀧先生が日本語で解説
\end{itemize}
 \item アハメド・アシュラフ先生「英語クラス」(月水金 5時限,金7時限)
\begin{itemize}
 \item 水 5時限に菊池先生が日本語で解説
\end{itemize}
\end{itemize}

\section{注意事項}

\begin{itemize}
 \item 原則的に指定されたクラスの授業をとってもらいます.
 \item 「基礎数学」は「微分積分学」の事前履修条件になっているため,
\underline{ダイナミックシラバスから通常の方法で履修申告できません}.
「微分積分学」に割り当てられた学生はかならず初回授業に出席し,
担当教員の支持に従ってください.
 \item 「基礎数学特別クラス」の学生が通常の「基礎数学」に移りたい場合や,
「線形代数英語クラス」の学生が通常の「線形代数」
に移りたい場合は,\underline{最初の授業時間に指定されたクラスに出席}して,
担当教員に相談してください.
\begin{itemize}
 \item 「基礎数学」は田澤先生クラスを推奨.
 \item 「線形代数」は瀧先生クラスを推奨.
\end{itemize}

\end{itemize}

\section{注意事項}

\begin{itemize}
 \item 数学科目は英語科目とリンクしています.
受講時限が変わる場合は英語担当の先生にも必ず連絡してください.
 \item 編入学生は単位互換等について学級担任に相談してください.
 \item その他何かわからないことがあれば
\underline{佐藤(研究棟501教員室)}に相談してください.
\end{itemize}

\end{document}